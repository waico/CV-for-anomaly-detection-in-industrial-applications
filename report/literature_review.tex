\section{LITERATURE REVIEW}
\label{LITERATURE REVIEW}

\subsection{Pipelines defects diagnostics}

Magnetic Flux Leakage (MFL) technique is the most common approach for oil and gas pipelines nondestructive diagnostics.
The data obtained in the pipeline inspection process is primarily analyzed by traditional machine learning (ML) methods.
A comparison of performance among different ML methods for defects identification problem is presented in \cite{Khodayari-Rostamabad2009}.
The main challenge for this approach is creating informative and important features that will be used as an input for ML methods.
Usually, these diagnostics features are generated using expert knowledge and manually-created heuristics.
It imposes the limitation on defects detection problem solving quality.
A variety of most successful features is presented and analyzed in details in \cite{Slesarev2017}.

To automate the process of feature generation Convolutional Neural Networks (CNNs) can be used.
As an advantage, they can solve the defects detection task and at the same time.
In literature there are samples of applying CNNs for defects detection \cite{Feng2017}, welds defect detection \cite{2020a}, welds detection and defects detection and classification \cite{Yang2020}, defect size estimation \cite{Lu2019}.
For all mentioned applications, CNNs outperformed existing traditional approaches.
Since DL relatively recently showed great progress in their results achievements, there are just a few works dedicated to MFL data analysis using DL.
For instance, we could not find any works on applying CNNs to defects segmentation task, despite the importance of this problem solving according to \cite{Feng2017}.
In this work, we want to solve two different tasks:
\begin{enumerate}
	\item Defects detection (Picture classification task).
	\item Defects segmentation (Instance segmentation task).
\end{enumerate}
