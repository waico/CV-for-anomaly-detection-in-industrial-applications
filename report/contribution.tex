\section{CONTRIBUTION}
\label{CONTRIBUTION}

\paragraph{Viacheslav Kozitsin}
The main task was to solve problem of semantic segmentation of defects in pipes.  Outcomes: A baseline was chosen as a stripped-down version of UNet (Figure \ref{ris:UNet}). The specifics of applying UNet under conditions of small sizes of the original image (64x64) were investigated. As a result, the problem was solved with IoU=0.2 and good visual results (Figure\ref{ris:mainresult}). Concomitant successes: annotated the data for the defect segmentation of pipe (Table \ref{tab:alg1}); proposed own approach to the preprocessing of raw data from sensors based on expert knowledge (Figure \ref{ris:preproc_fun}); solved the problem of displaced origins between data and report files (Figure\ref{ris:prepr}). Active participation in the filling of the report and presentation. 

\paragraph{Iurii Katser}
The main task was to solve the defects detection problem for the oil pipeline dataset.
Exploratory Data Analysis (EDA) was provided.
EDA allowed us to define issues in data that impede CV problems solving without data preprocessing.
Data preprocessing tools were implemented (abnormal values filling, defects, and welds coordinates searching).
The influence of different preprocessing algorithms was analyzed and presented in the results section.
The CNN that overperformed the best existing network on the presented pipelines dataset was proposed.
Outcomes and future work were marked.
	
\paragraph{Arman Alahyari}
	
\paragraph{Anton Hinneck}
	
\paragraph{Rahim Tariverdi}