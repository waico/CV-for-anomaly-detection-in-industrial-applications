\section{RESULTS}
\label{RESULTS}

We present the results of comparison for different preprocessing techniques and different CNN architectures for binary classification (normal pipe wall or defect/weld) in Tab.~\ref{tab:comp1} and multiclass classification problem (normal pipe wall, defect or weld) in Tab.~\ref{tab:comp2}.

Batch size is equal to 64, so the input to the network has shape (64, 1, 64, 64).
For all experiments, we use Adam optimizer with initial learning rate 0.001 and learning rate scheduler with parameters: threshold = 0.0001, factor = 0.5, min lr = 0.0001, patience = 484.
Also, for all experiments, the number of epochs is equal to 12.
Dropout rate for all experiments is equal to 0.33.
All mentioned parameters were selected by using grid search procedure.

\begin{table}[!htb]
	\caption{\label{tab:comp1}Comparison of performance among different classification methods for binary classification problem. $y=0$ - healthy; $y=1$ - defect/weld.}
	\begin{center}
		\small
		\begin{tabular}{| l | c | c | c |}
			\hline
			Method & $\hat{y}=y=0$ & $\hat{y}=y=1$ & Average \\
			\hline
			CNN-2 & 95.55 & 82.08 & 89.88 \\
			RayNet &  &  &  \\
			CNN-5 & 97.95 & \textbf{91.51} & \textbf{95.24} \\
			CNN-5+LRN & \textbf{98.29} & 89.86 & 94.74 \\
			\hline
			\multicolumn{4}{|c|}{Filling techniques comparison}  \\
			\hline
			CNN-5 (filling 1) & 97.95 & \textbf{91.51} & \textbf{95.24} \\
			CNN-5 (filling 2) & 97.95 & 84.20 & 92.16 \\
			CNN-5 (filling 3) & 97.26 & 83.02 & 91.27 \\
			CNN-5 (filling 4) & \textbf{98.63} & 81.13 & 91.27 \\
			CNN-5 (filling 5) & 98.12 & 81.84 & 91.27 \\
%			\hline
%			\multicolumn{4}{|c|}{Centering influence for the first filling method} \\
%			\hline
%			CNN-5 (centered) & 97.95 & \textbf{91.51} & 95.24 \\
%			CNN-5 (not centered) & \textbf{98.46} & 91.27 & \textbf{95.44} \\
%			CNN-2 (centered) & 95.55 & 82.08 & 89.88 \\
%			CNN-2 (not centered) & 96.92 & 80.42 & 89.81 \\
			\hline
		\end{tabular}
	\end{center}
\end{table}

Filling methods were researched for binary classification problem.
Centering means using peaks (extremums) searching procedure for welds or defects correct coordinates defining.
The centering procedure was research both for binary and multiclass classification problems.
Moreover, Min-Max normalization with using either a single image or whole dataset was investigated.
Finally, CNN-2 and CNN-5 were compared for centered images with the first filling method using single image Min-Max normalization.

\begin{table}[!htb]
	\caption{\label{tab:comp2}Comparison of performance among different classification methods for multiclass classification problem. $y=0$ - healthy; $y=1$ - defect; $y=2$ - weld.}
	\begin{center}
		\small
		\begin{tabular}{| l | c | c | c | c |}
			\hline
			Method & $\hat{y}=y=0$ & $\hat{y}=y=1$ & $\hat{y}=y=2$ & Average \\
			\hline
			CNN-2 & 97.60 & 59.86 & 92.91 & 90.97 \\
			RayNet &  &  &  &  \\
			CNN-5 & \textbf{98.12} & \textbf{76.76} & \textbf{98.23} & \textbf{95.14} \\
%			\hline
%			\multicolumn{5}{|c|}{Centering influence for the first filling method}  \\
%			\hline
%			CNN-5 (centered) & \textbf{98.12} & \textbf{85.21} & 75.18 & 89.88 \\
%			CNN-5 (not centered) &  \textbf{98.12} & 76.76 & \textbf{98.23} & \textbf{95.14} \\
%			CNN-2 (centered) & 96.75 & 71.13 & 52.13 & 80.65 \\
%			CNN-2 (not centered) & 97.60 & 59.86 & 92.91 & 90.97 \\
			\hline
			\multicolumn{5}{|c|}{Single image normalization vs Whole dataset normalization}  \\
			\hline
			CNN-5 (1) (whole) & 97.95 & 64.08 & \textbf{99.65} & 93.65 \\
			CNN-5 (1) (image) &  98.12 & \textbf{76.76} & 98.23 & \textbf{95.14} \\
			CNN-2 (1) (whole) & \textbf{99.32} & 13.38 & 96.45 & 86.41 \\
			CNN-2 (1) (image) & 97.60 & 59.86 & 92.91 & 90.97 \\
			CNN-5 (3) (whole) & 99.66 & 81.69 & 99.65 & 97.12 \\
%			CNN-5 (3) (image) &  98.12 & \textbf{76.76} & 98.23 & \textbf{95.14} \\
			CNN-2 (3) (whole) & 95.72 & 13.38 & 97.52 & 89.58 \\
%			CNN-2 (3) (image) & 97.60 & 59.86 & 92.91 & 90.97 \\
			\hline
		\end{tabular}
	\end{center}
\end{table}
